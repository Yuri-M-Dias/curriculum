% Portuguese files for translating the CV

\def \country{Brasil}
% Name and contact
\def \education{Educação}

% MS
\def \educationMSName{Me. Engenharia e Tecnologia\newline Espaciais}
\def \educationMSLocation{Instituto Nacional de Pesquisas Espaciais}
\def \educationMSDuration{2018 - Em andamento |\newline São José dos Campos}
\def \educationMSArea{Área: Engenharia e Gerenciamento de Sistemas Espaciais (CSE)}

% BS
\def \educationBSName{Engenharia de Software}
\def \educationBSLocation{Universidade Federal de Goiás (UFG)}
\def \educationBSDuration{2013-2016 | Goiânia}

% Skills
\def \skills{Conhecimentos}
\def \software{Software}
\def \softwareProfessional{Experiência Profissional}
\def \softwareProjects{Projetos}
\def \softwareHobby{Hobby}

% Languages
\def \languages{Línguas}
\def \languagesEN{Inglês: Fluente, TOEFL IBT 107 @ 2017}
\def \languagesFR{Francês: Intermediário}
\def \languagesJP{Japonês: JLPT N5 @ 2019}
\def \languagesPT{\smallskip}

% Coursework
\def \coursework{Matérias}
\def \courseworkSub{Sistemas Espaciais}
\def \courses{
Introdução a Comunicação via Satélite \\
Conceitos e Sistemas para Operação de Satélites \\
Sistemas de Dados Espaciais \\
Algoritmos de Data Science
}

% Professional Experience
\def \professional{Experiência Profissional}
\def \professionalSWE{Engenheiro de Software}
\def \professionalSWEI{Estágio em Engenharia de Software}
\def \professionalMLDur{Jul 2017 - Jan 2018 | Goiânia, BR}
\def \professionalMLDesc{
\begin{tightemize}
\item Pesquisei uma experiência em realidade aumentada (AR) para o museu Casa de Cora Coralina.
\item Mantive aplicação em Meteor para treinamento do RH da UFG relacionado à conduta ética, utilizando gamificação.
\end{tightemize}
}
\def \professionalFTDur{Abr 2017 - Jun 2017 | Remoto, sede na Noruega}
\def \professionalFTDesc{
\begin{tightemize}
\item Liderei esforços para introduzir práticas TDD e uma estrutura apropriada de testes e integração para a aplicação.
\item Parte do time full-stack em Ruby on Rails e Angular.
\end{tightemize}
}
\def \professionalZGDur{Set 2016 – Fev 2017 | Goiânia, BR}
\def \professionalZGDesc{
\begin{tightemize}
\item Parte de um dos times da aplicação principal, trabalhando full-stack com Grails.
\item Liderei uma tentativa inicial de adicionar testes automatizados com Cucumber, Selenium e Docker, com descrição natural utilizando BDD.
\end{tightemize}
}
\def \professionalINDur{Ago 2015 – Ago 2016 | Goiânia, BR}
\def \professionalINDesc{
\begin{tightemize}
\item Trabalhei re-estruturando a aplicação legado para utilizar novas tecnologias e abstrações, na época Spring e Java 8.
\item Modelagem dos relacionamentos de negócios em UML com o Astah, utilizandos pelo BD e ORM.
\end{tightemize}
}

% Research
\def \research{Pesquisa}
\def \researchMS{Mestrado}
\def \researchMSDur{Mar 2018 – Em andamento | São José dos Campos}
\def \researchMSDesc{
Trabalhando com ferramentas de análise de dados e algoritmos para fornecer suporte aos operadores do Centro de Controle de Satélites (CCS), com pesquisa focada em Big Data para Operação de Satélites.
Bolsista da CAPES.
}

% Publications
\def \publications{Publicações}
