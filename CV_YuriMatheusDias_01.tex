%%%%%%%%%%%%%%%%%%%%%%%%%%%%%%%%%%%%%%%%%
% Plasmati Graduate CV
% LaTeX Template
% Version 1.0 (24/3/13)
%
% This template has been downloaded from:
% http://www.LaTeXTemplates.com
%
% Original author:
% Alessandro Plasmati (alessandro.plasmati@gmail.com)
%
% License:
% CC BY-NC-SA 3.0 (http://creativecommons.org/licenses/by-nc-sa/3.0/)
%
% Important note:
% This template needs to be compiled with XeLaTeX.
% The main document font is called Fontin and can be downloaded for free
% from here: http://www.exljbris.com/fontin.html
%
%%%%%%%%%%%%%%%%%%%%%%%%%%%%%%%%%%%%%%%%%

%----------------------------------------------------------------------------------------
%	PACKAGES AND OTHER DOCUMENT CONFIGURATIONS
%----------------------------------------------------------------------------------------

\documentclass[a4paper,10pt]{article} % Default font size and paper size

\usepackage{fontspec} % For loading fonts
\defaultfontfeatures{Mapping=tex-text}
\setmainfont[SmallCapsFont = Fontin SmallCaps]{Fontin} % Main document font

\usepackage{xunicode,xltxtra,url,parskip} % Formatting packages

\usepackage[usenames,dvipsnames]{xcolor} % Required for specifying custom colors

\usepackage[big]{layaureo} % Margin formatting of the A4 page, an alternative to layaureo can be \usepackage{fullpage}
% To reduce the height of the top margin uncomment: \addtolength{\voffset}{-1.3cm}

\usepackage{hyperref} % Required for adding links	and customizing them
\definecolor{linkcolour}{rgb}{0,0.2,0.6} % Link color
\hypersetup{colorlinks,breaklinks,urlcolor=linkcolour,linkcolor=linkcolour} % Set link colors throughout the document

\usepackage{titlesec} % Used to customize the \section command
\titleformat{\section}{\Large\scshape\raggedright}{}{0em}{}[\titlerule] % Text formatting of sections
\titlespacing{\section}{0pt}{3pt}{3pt} % Spacing around sections

\usepackage{fontawesome} % Fontawesome for icons

\begin{document}

\pagestyle{empty} % Removes page numbering

\font\fb=''[cmr10]'' % Change the font of the \LaTeX command under the skills section

%----------------------------------------------------------------------------------------
%	NAME AND CONTACT INFORMATION
%----------------------------------------------------------------------------------------

\par{\centering{\Huge Yuri Matheus Dias Pereira}\bigskip\par} % Your name

\section{Dados pessoais}

\begin{tabular}{rl}
	\href{mailto:yurimathe.yp@gmail.com}{yurimathe.yp@gmail.com}
	& \faGithub
	\href{https://github.com/Yuri-M-Dias}{\bf https://github.com/Yuri-M-Dias} \\
	\faStackOverflow \href{http://stackoverflow.com/users/3312701/yuri-m-dias}
	{\bf StackOverflow}
	& \href{https://github.com/Yuri-M-Dias}{\bf Yuri-M-Dias} \\
	& Desenvolvedor Júnior/Engenheiro de Software. \\
\end{tabular}

%----------------------------------------------------------------------------------------
%	WORK EXPERIENCE
%----------------------------------------------------------------------------------------

\section{Formação Acadêmica}

\begin{tabular}{r|p{11cm}}
	\emph{Bacharelado} & Engenharia de Software \\
	& \emph{Instituto de Informática} da \emph{Universidade Federal de Goiás} \\
	& Término: 2016\\
\end{tabular}

%----------------------------------------------------------------------------------------
%	EDUCATION
%----------------------------------------------------------------------------------------

\section{Experiência com Desenvolvimento}

\begin{tabular}{rl}
	\textsc{Agosto} 2013 - & Bolsa Jovens Talentos CNPQ\\
	\textsc{Julho} 2014 & Desenvolvimento de uma aplicação LabWiki\\
	& Desenvolvimento de uma aplicação LabWiki para a aplicação LABORA. \\
	& Orientado pela professora Dra. Sand Luz Correa. Feito utilizando \\
	& Ruby on Rails, JavaScript, HTML e CSS. O objetivo da aplicação  \\
	& era a geração de um script em OEDL automaticamente, dada a \\
	& orientação dos nodos disponibilizados na página, com o seu ip. \\
	& \href{https://github.com/Yuri-M-Dias/LabWiki}{\bf Aplicação postada no GitHub} \\
	& \\
	\textsc{Agosto} 2013 - & Outros projetos desenvolvidos para a faculdade \\
	\textsc{Julho} 2015 &
	\href{https://bitbucket.org/Yuri-M-Dias/projeto-3/overview}
	{\bf Concorrência com SQLite em C usando API do UNIX.} \\
	& \href{https://github.com/Yuri-M-Dias/JavaBattleship}{\bf Jogo de Batalha Naval feito em Java com Design Patterns.}

\end{tabular}



%----------------------------------------------------------------------------------------
%	LANGUAGES
%----------------------------------------------------------------------------------------

\section{Idiomas}

\begin{tabular}{rl}
	\textsc{Inglês:} & Fluente, formação pelo You Get It!\\
	\textsc{Francês:} & Conhecimento básico, cursei até o Francês III no Centro de Línguas UFG\\
\end{tabular}

%----------------------------------------------------------------------------------------
%	COMPUTER SKILLS
%----------------------------------------------------------------------------------------

\section{Qualificações}

\begin{tabular}{rl}
	\textsc{Conhecimento básico:} & PHP, CSS, \LaTeX\\

	Conhecimento intermediário: & Java Standard Edition, Python 3.x, Ruby, C\\
	& JavaScript e HTML5\\

	Banco de Dados: & PostgreSQL, MySQL e SQLite. \\

	IDEs: & Eclipse, Sublime Text 2 e Spyder. \\

	Frameworks: & Ruby on Rails, JQuery, Bootstrap e Flask. \\

\end{tabular}

%----------------------------------------------------------------------------------------
%	SCHOLARSHIPS AND ADDITIONAL INFO
%----------------------------------------------------------------------------------------

\section{Atividades complementares}

\begin{tabular}{rl}
	2013 & $10^{\circ}$ Fórum Goiânia de Software Live(FGSL) e \\
	& $1^{a}$ Escola Regional de Informática de Goiás(ERI-GO)\\

	2014 & $11^{\circ}$ Fórum Goiânia de Software Live(FGSL) e \\
	& $2^{a}$ Escola Regional de Informática de Goiás(ERI-GO)\\

	2014 & Google Developers Group (GDG) DevFest Goiânia 2014\\
\end{tabular}


\end{document}
